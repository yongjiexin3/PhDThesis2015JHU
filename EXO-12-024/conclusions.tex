\section{Conclusions}
\label{sec:conclusions1}


An inclusive sample of multijet events corresponding to an integrated
luminosity of \intlumi, collected in pp collisions at
$\sqrt{s}=8$\TeVcc with the CMS detector, is used to measure the
$\PW/\cPZ$-tagged dijet mass spectrum for the two leading jets,
produced within the pseudorapidity range $|{\eta}| < 2.5$ with a
separation in pseudorapidity of $|{\Delta\eta}| < 1.3$. The generic
multijet background is suppressed using jet-substructure tagging
techniques that identify vector bosons decaying into $\qqbarprime$ pairs
merged into a single jet. In particular, the invariant mass of pruned
jets and the $N$-subjettiness ratio $\tau_{21}$ of each jet are used
to reduce the initially overwhelming multijet background. The
remaining background is estimated through a fit to smooth analytic
functions. With no evidence for a peak on top of the smoothly falling
background, lower limits are set at the 95\% confidence level on
masses of excited quark resonances decaying into qW and qZ at 3.2 and
2.9\TeVcc, respectively. Randall--Sundrum gravitons \GRS decaying into
WW are excluded up to 1.2\TeVcc, and $\PWpr$ bosons decaying into
$\PW\cPZ$, for masses less than 1.7\TeVcc.  For the first time mass
limits are set on $\PWpr\to \PW\cPZ$ and \GRS $\to \PW\PW$ in the
all-jets final state. The mass limits on $q^*\to q\PW$, $q^*\to
q\cPZ$, $\PWpr\to \PW\cPZ$, \GRS$\to \PW\PW$ are the most stringent
to date.  A model with a ``bulk'' graviton $\GBulk$ that decays into
WW or ZZ bosons is also studied, but no mass limits could be set due
to the small predicted cross sections.

