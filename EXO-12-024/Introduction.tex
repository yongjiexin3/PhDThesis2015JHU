\chapter{EXO-12-024}
\section{Introduction}
\label{sec:introduction}

Several scenarios of physics beyond the standard model predict the
existence of resonances with masses above $1$~\TeVcc which decay into a
quark jet and a \PW\ or \cPZ\ boson, or into a pair of \PW\ or \cPZ\
bosons.  A vector boson emerging from such a decay is usually
sufficiently boosted that the hadronization products from its daughter
quarks merge into a single massive jet~\cite{Gouzevitch:2013qca}.  We
present a search for these special dijet topologies, performed in the
pp~collision data collected by the CMS experiment
in 2012 at a center-of-mass energy of 8~TeV.

The signal is characterized by a peak in the dijet invariant mass,
emerging from the falling standard model background comprised mainly
of QCD events with a dijet topology. The analysis presented here
enhances the sensitivity to processes with jets from $\PW/\cPZ$ bosons
by the application of techniques that can identify $\PW/\cPZ$-jets and
suppress quark and gluon jets (``$\PW/\cPZ$-tagging'').  This search
is an update of a previous study~\cite{ref_2011} performed on the
7~TeV CMS data. Besides a larger dataset and an increased signal cross
section due to a higher center-of-mass energy, this update also
benefits from an improved $\PW/\cPZ$-tagger based on
``N-subjettiness'' variables~\cite{Thaler:2010tr}.

We consider four benchmark scenarios that would produce singly- or
doubly-tagged events: an excited quark $\cPq^*$~\cite{ref_qstar}
decaying into a quark and a W or Z boson; a Randall--Sundrum (RS)
graviton \GRS decaying to WW or ZZ ~\cite{rs1}; a Bulk graviton \GBulk
decaying to WW or ZZ~\cite{GravitonWWZZ1,GravitonWWZZ2,GravitonWWZZ3};
and a heavy partner of the SM W boson $\PWpr$ which decays to
WZ~\cite{ref_gauge}.  The most stringent limits on the $\cPq^*$ model
have been set in dijet resonance searches at the LHC by considering
inclusive all-hadronic final
states~\cite{exo12016,ATLASexcitedPAS,Harris:2011bh}.
The most stringent lower limit (at
95\% CL) on the $\cPq^*$ mass to date is 3.47\TeVcc~\cite{exo12016}.
Specific searches for the $\cPq\PW$ and $\cPq\cPZ$ final states have
previously been reported at the
Tevatron~\cite{CDFexcitedPAPER,D0excitedPAPER}, which exclude
resonances decaying to $\cPq\PW$ or $\cPq\cPZ$ with masses up to 0.54
\TeVcc, and at the LHC~\cite{ref_2011,CMSqZPAS}, which extends the
mass exclusion of $\cPq\PW$($\cPq\cPZ$) to 2.38\TeVcc(2.15\TeVcc).
The WW and ZZ final states have also been explored
experimentally~\cite{CMSZZPAS2,ATLASWWPAPER,ATLASZZPAPER,CDFZZPAPER},
setting lower limits on the \GRS and \GBulk mass as a function of the
coupling parameter $k/\MPl$, where $k$ determines the curvature of the
warped space and $\MPl$ is the reduced Planck mass ($\MPl \equiv
M_\text{Pl}/\sqrt{8\pi}$).  The \GBulk and \GRS models differ in the
fact that the \GBulk favors the decay into vector bosons rather than
photons or fermions and favors the production of longitudinal
polarized W or Z bosons.  For the $\PWpr$, the most stringent limits
are reported in searches with leptonic final
states~\cite{CMSwprimePAPER2013,ATLASwprimePAPER}, and the current
lower limit on the $\PWpr$ mass is 2.9\TeVcc.  The limit varies by
0.1\TeVcc, depending on the chirality of the $\PWpr$ couplings.
Specific searches in the WZ final state have also been
reported~\cite{CMSwprimeWZPAS,ATLASWWPAPER,ATLASwprimeWZPAS} setting a
lower limit of 1.1\TeVcc .

%\ifnpas
%Table~\ref{table:model} summarizes some properties of these models,
%and the published lower limits~\cite{CMSexcitedPAPER,ATLASexcitedPAPER,CMSexcitedPAS} on the mass of these 
%models in the dijet channel are shown in the last column.
%\begin{table}[htb]
%\begin{center}
%\begin{tabular}{ |l|c|r|r|r|r|r| }
%\hline
%Model Name & X & Color & J$^{p}$ & width/(2M) & Chan & lower limmits \\
%\hline
%Excited Quark & q* & Triplet & 1/2$^{+}$ &0.02 &qg& 3.2 \\
%RS Graviton &G & Singlet& 2$^{+}$& 0.01& gg, \qq&- \\
%Heavy W &W'& Singlet&1$^{-}$ & 0.01&\qq & 1.51\\
%\hline
%\end{tabular}
%\end{center}
%\caption{Properties of the resonance models we used and their lower limits~\cite{CMSexcitedPAPER,ATLASexcitedPAPER,CMSexcitedPAS}.}
%\label{table:model}
%\end{table}
%\fi

This analysis is the reload of the previous analysis EXO-11-095~\cite{ref_2011}.  
This analysis proceeds via the following steps:
\begin{enumerate}
\item The search is performed in the dijet sample, using the same
      preselection as the standard search for resonances decaying to 
      dijets~\cite{cmsdijet}.

\item We identify events with substructure: in each jet which is a candidate
    to originate from merging of W or Z daughter jets
  \begin{itemize}
  \item we require a pruned jet mass cut, and
  \item an n-subjettiness cut
  \end{itemize}

\item After the full event selection, a potential signal would be characterized as
    a peak in the dijet invariant mass, on top of a falling background distribution.

\item We model the background distribution with a smoothly falling 
  analytical function.

\item Finally, in absense of a signal, we set the limits on the production cross section of
  models with qW, qZ, WW, WZ and ZZ final states using the $CL_s$ approach.

\end{enumerate}
