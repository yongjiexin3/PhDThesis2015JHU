\chapter{Search for $X \to qV~or~VV$ at LHC at $\sqrt{s}=8$~TeV}
\label{chap:chapter3}
%massive resonances in dijet systems containing jets tagged as
%W or Z boson decays in pp collisions at $\sqrt{s}=8$~TeV. }
\section{Introduction}
\label{sec:introduction}

As we mentioned in Chapter~\ref{label:chapIntro}, the SM is limited and couldn't
provide solutions for some important phenomena, for example, the existence of 
graviton, the three generations of quarks and leptons, etc. 
Several models of physics beyond the standard model (SM) predict the
existence of resonances with masses above 1~TeV~ that decay into a
quark and a \PW\ or \cPZ\ vector boson, or into two vector bosons. In
proton-proton (pp) collisions at the energies reached at the Large Hadron
Collider (LHC), vector bosons emerging from such decays usually would have
sufficiently large momenta so that the hadronization products of their
$\qqbarprime$ decays would merge into a single massive
jet~\cite{Gouzevitch:2013qca}. We present a search for events
containing one or two jets of this kind in pp collisions at a
centre-of-mass energy of $\sqrt{s}=8$~TeV.  The data sample,
corresponding to an integrated luminosity of 19.7~\fbinv, was collected
with the CMS detector at the LHC.

The signal is characterized by a peak in the dijet invariant mass
distribution $m_\mathrm{jj}$ over a continuous background from SM
processes, comprised mainly of multijet events from quantum
chromodynamic (QCD) processes. The sensitivity to jets from \PW\ or
\cPZ\ bosons is enhanced through the use of jet-substructure
techniques that help differentiate such jets from remnants of quarks
and gluons~\cite{topwtag_pas,JME-13-006}, providing the possibility of
``$\PW/\cPZ$-tagging''. This search is an update of a previous CMS
study~\cite{ref_2011} performed using data from pp collisions at
$\sqrt{s}=7$~TeV. Besides increased data-sample size and larger
signal cross sections from the increase in centre-of-mass energy, this
analysis also benefits from an improved $\PW/\cPZ$-tagger based on
``$N$-subjettiness'' variables, introduced in
Ref.~\cite{Thaler:2010tr} and defined in Section~\ref{sec:N-subjettiness1}.

We consider four reference processes that yield one $\PW/\cPZ$-tagged
or two $\PW/\cPZ$-tagged all-jet events: (i) an excited quark
q*~\cite{ref_qstar, ref_qstar2} that decays into a quark and
either a \PW\ or a \cPZ\ boson, (ii) a Randall--Sundrum (RS) graviton
\GRS~that~ decays into \PW \PW\ or \cPZ \cPZ\
bosons~\cite{rs1,Randall:1999vf}, (iii) a ``bulk'' graviton $\GBulk$
that decays into \PW \PW\ or \cPZ \cPZ\
~\cite{GravitonWWZZ1,GravitonWWZZ2,GravitonWWZZ3}, and (iv) a
heavy partner of the SM \PW\ boson (\PWpr) that decays into \PW\cPZ\
\cite{egm}.


%{\bf Excited quark}
%There are three generations of quarks and lepton in the SM, however, without 
%any explanation. Composite models of quarks and leptons with their potential to 
%explain the generation structure of quarks-leptons have been quite popular. 
%These excited fermions are expected to have resonance mass at least a few 
%hundred~GeV, according to present experimental measurements.
%LHC with its high collision energy, is a great farm to produce such 
%massive resonances. 

%At LHC, excited quarks can be produced either pairwise or singly. Pair production, 
%which mainly leads to four jets or jets plus W or Z vector bosons, has a small cross-section, 
%and a large QCD 
%4-jet background. While singly production, with q* decaying into one quark and one
%vector boson (W or Z) can be distinguished by the W/Z-tagging techniques from 
%the QCD 2-jet background. 

%{\bf Randall-Sundrum graviton}


Results from previous searches for these signal models include limits
placed on the production of q* at the LHC as
dijet~\cite{exo12016,ATLASexcitedPAS,Harris:2011bh} or
$\gamma$+jet~\cite{Aad:2013cva} resonances, with a q* lighter
than $\approx$3.5~TeV at a confidence level (CL) of
95\%~\cite{exo12016}. Specific searches for resonant qW and
qZ final states at the
Tevatron~\cite{CDFexcitedPAPER,D0excitedPAPER} exclude q* decays
into qW or qZ with $m_{q*}<0.54$~TeV, and results
from the LHC~\cite{ref_2011,CMSqZPAS} exclude q* decays into
qW or qZ for $m_{q*}< 2.4$~TeV and $m_{q*}<
2.2$~TeV, respectively.

Resonances in final states containing candidates for \PW\PW\ or
\cPZ\cPZ\ systems have also been
sought~\cite{CMSZZPAS2,ATLASWWPAPER,ATLASZZPAPER,CDFZZPAPER}, with
lower limits set on the masses of \GRS and $\GBulk$ as a function of
the coupling parameter $k/\MPl$, where $k$ reflects the curvature of
the warped space, and $\MPl$ is the reduced Planck mass ($\MPl \equiv
M_\text{Pl}/\sqrt{8\pi}$)~\cite{rs1,Randall:1999vf}. The bulk graviton
model is an extension of the original RS model that addresses the
flavour structure of the SM through localization of fermions in the
warped extra dimension. The experimental signatures of the \GRS and
$\GBulk$ models differ in that $\GBulk$ favours the production of
gravitons through gluon fusion, with a subsequent decay into vector bosons,
rather than production and decay through fermions or photons, as the
coupling to these is highly suppressed. As a consequence, $\GBulk$
preferentially produces \PW\ and \cPZ\ bosons that are longitudinally polarized,
while \GRS favours the production of transversely polarized
\PW\ or \cPZ\ bosons. In this study, we use an improved calculation of
the $\GBulk$ production cross section~\cite{GravitonWWZZ1} that predicts a factor of four smaller yield than
assumed in previous studies~\cite{CMSZZPAS2,ATLASWWPAPER}.

The most stringent limits on $\wpr$ boson production  are those reported
for searches in leptonic final
states~\cite{CMSwprimePAPER2013,ATLASwprimePAPER}, with the current
limit specified by $m_{\PWpr}>2.9$~TeV. Depending on the chirality
of the $\PWpr$ couplings, this limit could change by $\approx$0.1~TeV. Searches for $\PWpr$ in the \PW\cPZ\ channel have also been
reported~\cite{CMSwprimeWZPAS,ATLASWWPAPER,ATLASwprimeWZPAS} and set
a lower limit of $m_{\PWpr}>1.1$~TeV.

The data, and the event simulations are described briefly
in Section~\ref{sec:data_and_mc_samples1}. Event reconstruction, including
details of $\PW/\cPZ$-tagging, and selection criteria are discussed in
Section~\ref{sec:analysis1} and~\ref{sec: W/Z tagging1}. 
The systematic uncertainties are discussed in Section~\ref{sec:systematics1}. 
And Section~\ref{sec:background1} presents
studies of dijet mass spectra, including SM background estimates.
The interpretation of the results in terms of the benchmark signal models
is presented in Section~\ref{sec:results1}, and the results are summarized in Section~\ref{sec:conclusions1}.





