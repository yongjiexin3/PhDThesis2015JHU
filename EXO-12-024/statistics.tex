\newpage
\section{Limit setting procedure}
\label{sec:statistics}



We search for a peak on top of the falling background spectrum by
means of a maximum likelihood fit to the data. The likelihood $\mathcal{L}$, computed using events binned as a function of $m_\mathrm{jj}$,
is written as
\begin{equation} \mathcal{L} = \prod_{i}
  \frac{\lambda_{i}^{n_{i}}\re^{-\lambda_{i}}}{n_{i}!},
\end{equation}
where ${\lambda_{i}} = {\mu}{N_{i}(S)} + {N_{i}(B)}$,
$\mu$ is a scale factor for the signal, $N_i(S)$ is the number
expected from the signal, and $N_i(B)$ is the number expected from
multijet background. The parameter $n_i$ quantifies the number of
events in the $i^\mathrm{th}$ $m_\mathrm{jj}$ mass bin.
The background $N_i(B)$ is described by the functional form of
Equation~(\ref{eqParam1}). While maximizing the likelihood as a function of
the resonance mass, $\mu$ as well as the parameters of the background
function are left floating.


We quantify the consistency of the data with the null hypothesis as a
function of resonance mass for the benchmark models through the local
p-value. The largest local significance in the singly
$\PW/\cPZ$-tagged sample is observed for the hypothesis of a ${\rm q*
\to q\PW}$ resonance of mass 1.5\TeVcc, and is equivalent to an excess
of 1.8 standard deviations. The largest local significance in the
doubly tagged event sample corresponds to an excess of 1.3 standard
deviations for a  \GRS$\to\PW\PW$ resonance of mass 1.9\TeVcc. Using the
${\rm \GBulk\to\PW\PW/\cPZ\cPZ}$ model, where the LP and HP categories
contribute in different proportions compared to the case for the
\GRS$\to\PW\PW$ model, yields no excess larger than one standard
deviation. 

Using pseudo-experiments, we estimated the probability of observing a
local statistical fluctuation of at least two standard deviations in
any mass bin. This probability corresponds to an equivalent global
significance of one standard deviation.  The $m_\mathrm{jj}$ distributions
are used to set upper limits on the product of the production cross
sections and decay branching fractions for the benchmark models.