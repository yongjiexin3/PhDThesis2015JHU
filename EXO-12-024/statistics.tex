\newpage
\section{Limit setting procedure}
\label{sec:statistics}

For setting upper limits on the resonance production cross section
the asymptotic approximation~\cite{AsymptCLs} of the LHC $CL_s$ method~\cite{CLs1,CLs2} is used.
The binned likelihood, $L$, can be written as:
\begin{equation}
L = \prod_{i} \frac{\mu_{i}^{n_{i}} \re^{-\mu_{i}}}{n_{i}!} \ ,
\end{equation}
where
\begin{equation}
{\mu_{i}} = {\sigma}{N_{i}(S)} + {N_{i}(B)} \ ,
\label{function}
\end{equation}
$n_i$ is the observed number of events in the $i^{th}$ dijet mass bin, and
$N_i(S)$ is the expected number of events from the signal in the $i^{th}$ dijet
mass bin, $\sigma$ scales the signal amplitude, and
$N_i(B)$ is the expected number of events from background in the
$i^{th}$ dijet mass bin.
The background $N_i(B)$ is estimated as the background component
of the best signal+background fit to the data points.
The dominant sources of systematic uncertainties (the jet energy
scale, the jet energy resolution, the integrated luminosity, and the
$\PW/\cPZ$-tagging efficiency) are considered as nuisance parameters associated to log-normal priors.
The dependence of the likelihood on their
value is removed through profiling. The ratio of the profiled
likelihood at a given value $\sigma^*$ over the maximum of the
likelihood (for $0<\sigma<\sigma^*$) is used as a test statistics
to compute the $CL_s$ value associated to $\sigma^*$. This allows to
determine the $95\%$ conficence-level (CL) limit.
