\iffalse

The full event selection efficiency is estimated using simulated
signal samples.
Less than 1\% of the $\cPZ\cPZ$ or $\PW\PW$ events which pass the full
selection are from $\cPZ\cPZ \to ll qq$ or $\PW\PW \to l\nu qq$
decays, where $l$ can be a muon or electron.  While 3\% of the
selected $\cPZ\cPZ$ events are from $\cPZ\cPZ \to \tau\tau qq$ decays,
less than 1\% of the selected $\PW\PW$ events are from $\PW\PW \to
\tau\nu qq$ decays.
To within 10\% accuracy the full selection efficiency can
therefore be
approximated by the product of the $\PW/\cPZ$-tagging efficiency
and an approximate acceptance.
This acceptance is shown in Fig.~\ref{fig:acceptances} and
takes into account the angular acceptance
($|\eta| < 2.5$, $|\Delta\eta|<1.3$),
the branching into quark final states,
BR($\PW/\cPZ \to \text{quarks}$) and a matching within
$\Delta R = \sqrt{(\Delta \eta)^2 + (\Delta\phi)^2} <0.5$
between the generated W/Z bosons and the reconstructed jets.


%\begin{table}[htbp]
%\begin{tabular}{|l|l|r|r|l|l|r|r|}
%\hline
%signal  & model & mass & BR($W/Z \to jets$)$\times$acc. & M 1-tag & H 1-tag & M 2-tag & H 2-tag \\ \hline
%$G_{RS} \to WW$ & Herwig++ & 1000 & 0.382 & NAN & NAN & 0.073 & 0.116 \\ 
%$G_{RS} \to WW$ & Herwig++ & 1500 & 0.401 & NAN & NAN & 0.076 & 0.092 \\ 
%$G_{RS} \to WW$ & Herwig++ & 1800 & 0.403 & NAN & NAN & 0.082 & 0.089 \\ 
%$G_{RS} \to WW$ & Herwig++ & 2000 & 0.404 & NAN & NAN & 0.076 & 0.079 \\ 
%$G_{RS} \to WW$ & Herwig++ & 2500 & 0.411 & NAN & NAN & 0.055 & 0.046 \\ 
%$G_{RS} \to WW$ & Herwig++ & 3000 & 0.409 & NAN & NAN & 0.032 & 0.029 \\ \hline 
%$G_{RS} \to ZZ$ & Herwig++ & 1000 & 0.390 & NAN & NAN & 0.070 & 0.120 \\ 
%$G_{RS} \to ZZ$ & Herwig++ & 1500 & 0.391 & NAN & NAN & 0.088 & 0.119 \\ 
%$G_{RS} \to ZZ$ & Herwig++ & 1800 & 0.387 & NAN & NAN & 0.095 & 0.125 \\ 
%$G_{RS} \to ZZ$ & Herwig++ & 2000 & 0.383 & NAN & NAN & 0.095 & 0.121 \\ 
%$G_{RS} \to ZZ$ & Herwig++ & 3000 & 0.374 & NAN & NAN & 0.077 & 0.056 \\ \hline
%$G_{RS} \to WW$ & Pythia Z2* & 1000 & 0.455 & NAN & NAN & 0.081 & 0.132 \\ 
%$G_{RS} \to WW$ & Pythia Z2* & 1500 & 0.432 & NAN & NAN & 0.088 & 0.124 \\ 
%$G_{RS} \to WW$ & Pythia Z2* & 1800 & 0.419 & NAN & NAN & 0.079 & 0.109 \\ 
%$G_{RS} \to WW$ & Pythia Z2* & 2000 & 0.415 & NAN & NAN & 0.078 & 0.094 \\ 
%$G_{RS} \to WW$ & Pythia Z2* & 2500 & 0.397 & NAN & NAN & 0.059 & 0.056 \\ 
%$G_{RS} \to WW$ & Pythia Z2* & 3000 & 0.391 & NAN & NAN & 0.040 & 0.035 \\ \hline
%$G_{RS} \to ZZ$ & Pythia Z2* & 1000 & 0.454 & NAN & NAN & 0.075 & 0.149 \\ 
%$G_{RS} \to ZZ$ & Pythia Z2* & 1500 & 0.417 & NAN & NAN & 0.106 & 0.161 \\ 
%$G_{RS} \to ZZ$ & Pythia Z2* & 1800 & 0.400 & NAN & NAN & 0.107 & 0.154 \\ 
%$G_{RS} \to ZZ$ & Pythia Z2* & 2000 & 0.389 & NAN & NAN & 0.113 & 0.148 \\ 
%$G_{RS} \to ZZ$ & Pythia Z2* & 3000 & 0.355 & NAN & NAN & 0.081 & 0.071 \\ \hline
%$W' \to WZ$ & Pythia Z2* & 1000 & 0.486 & NAN & NAN & 0.109 & 0.246 \\ 
%$W' \to WZ$ & Pythia Z2* & 1500 & 0.487 & NAN & NAN & 0.121 & 0.246 \\ 
%$W' \to WZ$ & Pythia Z2* & 1800 & 0.485 & NAN & NAN & 0.119 & 0.216 \\ 
%$W' \to WZ$ & Pythia Z2* & 2000 & 0.484 & NAN & NAN & 0.117 & 0.199 \\ 
%$W' \to WZ$ & Pythia Z2* & 2200 & 0.479 & NAN & NAN & 0.108 & 0.172 \\ 
%$W' \to WZ$ & Pythia Z2* & 2500 & 0.485 & NAN & NAN & 0.092 & 0.139 \\ 
%$W' \to WZ$ & Pythia Z2* & 3000 & 0.494 & NAN & NAN & 0.068 & 0.114 \\ \hline
%$q* \to qW$ & Pythia Z2* & 1000 & 0.502 & 0.125 & 0.364 & NAN & NAN \\ 
%$q* \to qW$ & Pythia Z2* & 1500 & 0.495 & 0.135 & 0.333 & NAN & NAN \\ 
%$q* \to qW$ & Pythia Z2* & 2000 & 0.497 & 0.139 & 0.306 & NAN & NAN \\ 
%$q* \to qW$ & Pythia Z2* & 3000 & 0.490 & 0.129 & 0.177 & NAN & NAN \\ 
%$q* \to qW$ & Pythia Z2* & 4000 & 0.475 & 0.121 & 0.130 & NAN & NAN \\ \hline
%$ q* \to qZ$ & Pythia Z2* & 1000 & 0.494 & 0.116 & 0.371 & NAN & NAN \\ 
%$ q* \to qZ$ & Pythia Z2* & 1500 & 0.488 & 0.138 & 0.384 & NAN & NAN \\ 
%$ q* \to qZ$ & Pythia Z2* & 2000 & 0.474 & 0.154 & 0.364 & NAN & NAN \\ 
%$ q* \to qZ$ & Pythia Z2* & 3000 & 0.452 & 0.173 & 0.262 & NAN & NAN \\ 
%$ q* \to qZ$ & Pythia Z2* & 4000 & 0.449 & 0.156 & 0.187 & NAN & NAN \\ \hline
%\end{tabular}
%\caption{Summary of signal branching ratio times angular acceptance and W/Z-tagging efficiency, in medium purity 1-tag, high purity 
%1-tag, medium purity 2-tag and high purity 2-tag categories. }
%\label{table:acceptance}
%\end{table}
%






%\begin{table}[htb]
%\begin{center}
%\begin{tabular}{ |c|c|c|c|c|c| }
%\hline
%signal & model & mass & BR($W/Z \to jets$)$\times$acc. & 1 W/Z-tag & 2 W/Z-tag\\
%\hline
%hline
%\end{tabular} 
%\end{center}
%\caption{Summary of signal branching ratio times acceptance and W/Z-tagging efficiency.}
%\label{table:acceptance}
%\end{table}

\begin{figure}[htb]
\begin{center}
\includegraphics[width=0.88\textwidth]{figs/signal-acc-eff/all-signal-acc-8TeV.pdf}
\end{center}
\caption{Fraction of events with branching into quark final states, BR($\PW/\cPZ \to \text{quarks}$),
  which are reconstructed as dijets ($\text{quarks} \to \text{jets}$)
  and pass the angular acceptance ($|\eta| < 2.5$, $|\Delta\eta|<1.3$.}
\label{fig:acceptances}
\end{figure}

\begin{figure}[htb]
\begin{center}
\includegraphics[width=0.49\textwidth]{figs/signal-acc-eff/single-tagging-eff-medium.pdf}
\includegraphics[width=0.49\textwidth]{figs/signal-acc-eff/single-tagging-eff.pdf}
\end{center}
\caption{The fraction of singly-tagged events, requiring one medium purity (left) 
  and high purity (right) $\PW/\cPZ$-tag in data,
  signal and background simulations for events passing the angular acceptance
  requirement ($|\eta| < 2.5$, $|\Delta\eta|<1.3$).}
\label{fig:singleefficiencies}
\end{figure}

\begin{figure}[htb]
\begin{center}
\includegraphics[width=0.49\textwidth]{figs/signal-acc-eff/double-tagging-eff-medium.pdf}
\includegraphics[width=0.49\textwidth]{figs/signal-acc-eff/double-tagging-eff.pdf}
\end{center}
\caption{The fraction of doubly-tagged events, requiring two medium purity (left) 
and high purity (right) $\PW/\cPZ$-tags in data,
  signal and background simulations for events passing the angular acceptance
  requirement ($|\eta| < 2.5$, $|\Delta\eta|<1.3$).}
\label{fig:doubleefficiencies}
\end{figure}
%


The W/Z-tagging efficiency is shown in Fig.~\ref{fig:singleefficiencies} and Fig.~\ref{fig:doubleefficiencies}.
%The W/Z-tagging efficiency for single W/Z-tagged signals is about $13\% - 38\%$ in high purity category, and $12\% - 18\%$ in medium purity category.
%The W/Z-tagging efficiency for a double W/Z-tagged signals is about $3\% - 12\%$ in high purity category, and $3\% - 9\%$ in medium purity category.


The signal shapes for all five processes considered in this analysis are shown in Fig.~\ref{fig:mediumsignalShapes} and Fig.~\ref{fig:highsignalShapes}.  
For the $qW$ and $qZ$ final states the shape with a single W/Z-tag required is shown, while for the other signals two W/Z-tags are required.


%Fig~\ref{fig:acceptanceGstarWWherwig} and Fig~\ref{fig:acceptanceGstarWWpythia} show the signal shape for process: $G_{RS} \to WW$.
%Fig~\ref{fig:acceptanceGstarZZherwig} and Fig~\ref{fig:acceptanceGstarZZpythia} show the signal shape for process: $G_{RS} \to ZZ$.  
%Fig~\ref{fig:acceptanceWprime} shows the signal shape for process: $W' \to WZ$. 
%Fig~\ref{fig:acceptanceqstarqW} and Fig~\ref{fig:acceptanceqstarqZ} shows the signal shape for process: $q* \to qW$, $qZ$.
%The different line colors correspond to different signal resonance masses.
%The dashed line is for signal with a single W/Z-tag, while the dotted 
%line is for signal with two W/Z-tags.

\begin{figure}[htb]
\begin{center}
\includegraphics[width=0.75\textwidth]{figs/signal-acc-eff/resonance-shape-medium.pdf}
\end{center}
\caption{The normalized medium purity signal resonance distribution for  $G_{RS}\to \wboson\wboson$, $G_{RS}\to \zboson\zboson$, $W' \to WZ$, $q*\to qW$, and $q*\to qZ$ resonances of dijet invariant mass 1.0\TeVcc, 1.5\TeVcc, 2.0\TeVcc, 2.5 \TeVcc,  3.0\TeVcc, 4.0\TeVcc.
}
\label{fig:mediumsignalShapes}
\end{figure}


\begin{figure}[htb]
\begin{center}
\includegraphics[width=0.75\textwidth]{figs/signal-acc-eff/resonance-shape.pdf}
\end{center}
\caption{The normalized high purity signal resonance distribution for  $G_{RS}\to \wboson\wboson$, $G_{RS}\to \zboson\zboson$, $W' \to WZ$, $q*\to qW$, and $q*\to qZ$ resonances of dijet invariant mass 1.0\TeVcc, 1.5\TeVcc, 2.0\TeVcc,  3.0\TeVcc, 4.0\TeVcc.
}
\label{fig:highsignalShapes}
\end{figure}

%\begin{figure}[htb]
%\begin{center}
%\includegraphics[width=0.75\textwidth]{figs/signal-acc-eff/resonance-shape-double-1500.pdf}
%\end{center}
%\caption{The normalized signal resonance distribution of Herwig++ and Pythia6 Z2* for 1.5 \TeVcc $G_{RS}\to \wboson\wboson$, $G_{RS}\to \zboson\zboson$ resonances of dijet invariant mass.
%}
%\label{fig:signalShapes2}
%\end{figure}




%\begin{figure}[htb]
%\begin{center}
%\includegraphics[width=0.48\textwidth]{figs/signal-acc-eff/GstarWWherwig.pdf}
%\includegraphics[width=0.48\textwidth]{figs/signal-acc-eff/RSGWWherwig-deta.pdf}
%\end{center}
%\caption{Signal shape in AK5 dijet mass and $\Delta \eta$: $G_{RS} \to WW$ using Herwig++.
%The shape in AK5 dijet mass is normalized to the number of generated events (with phasespace cuts).
%The distribution in $\Delta \eta$ is normalized to the number of events passing the analysis selection in the inclusive category (no W/Z-tagging required).} 
%\label{fig:acceptanceGstarWWherwig}
%\end{figure}


%\begin{figure}[htb]
%\begin{center}
%\includegraphics[width=0.48\textwidth]{figs/signal-acc-eff/GstarWWpythia.pdf}
%\includegraphics[width=0.48\textwidth]{figs/signal-acc-eff/RSGWWpythia-deta.pdf}
%\end{center}
%\caption{Signal shape in AK5 dijet mass and $\Delta \eta$: $G_{RS} \to WW$ using Pythia Z2*.
%The shape in AK5 dijet mass is normalized to the number of generated events (with phasespace cuts).
%The distribution in $\Delta \eta$ is normalized to the number of events passing the analysis selection in the inclusive category (no W/Z-tagging required).}
%\label{fig:acceptanceGstarWWpythia}
%\end{figure}

%\begin{figure}[htb]
%\begin{center}
%\includegraphics[width=0.48\textwidth]{figs/signal-acc-eff/GstarZZherwig.pdf}
%\includegraphics[width=0.48\textwidth]{figs/signal-acc-eff/RSGZZherwig-deta.pdf}
%\end{center}
%\caption{Signal shape in AK5 dijet mass and $\Delta \eta$: $G_{RS} \to ZZ$ using Herwig++.
%The shape in AK5 dijet mass is normalized to the number of generated events (with phasespace cuts).
%The distribution in $\Delta \eta$ is normalized to the number of events passing the analysis selection in the inclusive category (no W/Z-tagging required).}
%\label{fig:acceptanceGstarZZherwig}
%\end{figure}



%\begin{figure}[htb]
%\begin{center}
%\includegraphics[width=0.48\textwidth]{figs/signal-acc-eff/GstarZZpythia.pdf}
%\includegraphics[width=0.48\textwidth]{figs/signal-acc-eff/RSGZZpythia-deta.pdf}
%\end{center}
%\caption{Signal shape in AK5 dijet mass and $\Delta \eta$: $G_{RS} \to ZZ$ using Pythia Z2*.
%The shape in AK5 dijet mass is normalized to the number of generated events (with phasespace cuts).
%The distribution in $\Delta \eta$ is normalized to the number of events passing the analysis selection in the inclusive category (no W/Z-tagging required).}
%\label{fig:acceptanceGstarZZpythia}
%\end{figure}


%\begin{figure}[htb]
%\begin{center}
%\includegraphics[width=0.48\textwidth]{figs/signal-acc-eff/WZWprime.pdf}
%\includegraphics[width=0.48\textwidth]{figs/signal-acc-eff/Wprime-deta.pdf}
%\end{center}
%\caption{Signal shape in AK5 dijet mass and $\Delta \eta$: $W' \to WZ$.
%The shape in AK5 dijet mass is normalized to the number of generated events (with phasespace cuts).
%The distribution in $\Delta \eta$ is normalized to the number of events passing the analysis selection in the inclusive category (no W/Z-tagging required).}
%\label{fig:acceptanceWprime}
%\end{figure}

%\begin{figure}[htb]
%\begin{center}
%\includegraphics[width=0.48\textwidth]{figs/signal-acc-eff/qstarqw.pdf}
%\includegraphics[width=0.48\textwidth]{figs/signal-acc-eff/QstarToQW-deta.pdf}
%\end{center}
%\caption{Signal shape in AK5 dijet mass and $\Delta \eta$: $q*\to qW$.
%The shape in AK5 dijet mass is normalized to the number of generated events (with phasespace cuts).
%The distribution in $\Delta \eta$ is normalized to the number of events passing the analysis selection in the inclusive category (no W/Z-tagging required).}
%\label{fig:acceptanceqstarqW}
%\end{figure}

%\begin{figure}[htb]
%\begin{center}
%\includegraphics[width=0.48\textwidth]{figs/signal-acc-eff/qstarqz.pdf}
%\includegraphics[width=0.48\textwidth]{figs/signal-acc-eff/QstarToQZ-deta.pdf}
%\end{center}
%\caption{Signal shape in AK5 dijet mass and $\Delta \eta$: $q*\to qZ$.
%The shape in AK5 dijet mass is normalized to the number of generated events (with phasespace cuts).
%The distribution in $\Delta \eta$ is normalized to the number of events passing the analysis selection in the inclusive category (no W/Z-tagging required).}
%\label{fig:acceptanceqstarqZ}
%\end{figure}
\fi
