\newpage
\section{Limit setting procedure}
\label{sec:statistics}


We search for a peak on top of the falling background spectrum by
means of a maximum likelihood fit to the data. The likelihood $\mathcal{L}$, computed using events binned as a function of $m_\mathrm{jj}$,
is written as
\begin{equation} \mathcal{L} = \prod_{i}
  \frac{\lambda_{i}^{n_{i}} e^{-\lambda_{i}}}{n_{i}!},
\end{equation}
where ${\lambda_{i}} = {\mu}{N_{i}(S)} + {N_{i}(B)}$,
$\mu$ is a scale factor for the signal, $N_i(S)$ is the number
expected from the signal, and $N_i(B)$ is the number expected from
multijet background. The parameter $n_i$ quantifies the number of
events in the $i^\mathrm{th}$ $m_\mathrm{jj}$ mass bin.
The background $N_i(B)$ is described by the functional form of
Eq.~(\ref{eqParam}). While maximizing the likelihood as a function of
the resonance mass, $\mu$ as well as the parameters of the background
function are left floating.

The asymptotic approximation~\cite{AsymptCLs} of the LHC
$\mathrm{CL_s}$ method~\cite{CLs1,CLs3} is used to set upper limits on
the cross sections for resonance production. The dominant sources of
systematic uncertainties are treated as nuisance parameters associated
with log-normal priors in those variables.
% following the methodology
%described in Ref.~\cite{ATLASCMSstat}. 
For a given value of the
signal cross section, the nuisance parameters are fixed to the values
that maximize the likelihood, a method referred to as
profiling. The dependence of the likelihood on parameters used to
describe the background in Eq.~(\ref{eqParam}) is removed in the same
manner, and no additional systematic uncertainty is therefore assigned
to the parameterization of the background.



%For setting upper limits on the resonance production cross section
%the asymptotic approximation~\cite{AsymptCLs} of the LHC $CL_s$ method~\cite{CLs1,CLs2} is used.
%The binned likelihood, $L$, can be written as:
%\begin{equation}
%L = \prod_{i} \frac{\mu_{i}^{n_{i}} \re^{-\mu_{i}}}{n_{i}!} \ ,
%\end{equation}
%where
%\begin{equation}
%{\mu_{i}} = {\sigma}{N_{i}(S)} + {N_{i}(B)} \ ,
%\label{function}
%\end{equation}
%$n_i$ is the observed number of events in the $i^{th}$ dijet mass bin, and
%$N_i(S)$ is the expected number of events from the signal in the $i^{th}$ dijet
%mass bin, $\sigma$ scales the signal amplitude, and
%$N_i(B)$ is the expected number of events from background in the
%$i^{th}$ dijet mass bin.
%The background $N_i(B)$ is estimated as the background component
%of the best signal+background fit to the data points.
%The dominant sources of systematic uncertainties (the jet energy
%scale, the jet energy resolution, the integrated luminosity,
%and the b-tagging scale factor, 
%$H-\PW/\cPZ$-tagging efficiency) are 
%considered as nuisance parameters associated to log-normal priors.
%The dependence of the likelihood on their
%value is removed through profiling. The ratio of the profiled
%likelihood at a given value $\sigma^*$ over the maximum of the
%likelihood (for $0<\sigma<\sigma^*$) is used as a test statistics
%to compute the $CL_s$ value associated to $\sigma^*$. This allows to
%determine the $95\%$ conficence-level (CL) limit.
