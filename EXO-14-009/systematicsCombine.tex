
\section{Systematic uncertainties}
\label{sec:systematics2}


The largest contributions to 
the systematic uncertainty are 
associated with the modelling of the signal, 
namely: the efficiencies of $\PW/\cPZ$, H, and b tagging; 
the choice of PDF; the jet energy 
scale (JES); the jet energy resolution (JER); the 
pileup corrections; the cross-talk between 
different signal contributions; and the integrated luminosity.


The uncertainty in the efficiency for $\PW/\cPZ$-tagging 
is estimated using a control sample enriched with ${\rm t\bar{t}}$ 
events
described in Ref.~\cite{JME-13-006}. 
Uncertainties of\scalefactorHPu~and\scalefactorLPu~in
 the respective scale factors for HP and LP V tag
include contributions from control-sample statistical uncertainties,
and the uncertainties in the JES and JER for 
pruned jets~\cite{Khachatryan:2014hpa}.
To extrapolate to
higher jet \pt, %from the study of {\bf EXO-12-024 and JME-13-009}, 
an estimation of V tagging efficiency varying
as a function of
\pt for two different showering and hadronization models using
\PYTHIA~6 and \HERWIG{++}, 
shows that the differences are within $4\%$ ($12\%$)
for the HP (LP) V-tagging~\cite{JME-13-006}. 
%In this procedure, the \HERWIG{++} efficiency as a function
%of \pt is normalized so that
%that \PYTHIA~6 and \HERWIG{++} agree at low \pt, where the
%efficiency is measured in data, but are different at high \pt.
We extrapolate the $\Hww$ tagging efficiency
scale factor in the same way as the W/Z-tagging efficiency, with 
an additional systematic uncertainty based on the difference between
\PYTHIA~6 and \HERWIG{++} in modelling $\Hww$ decay. 
%Different from W/Z-tagging, 
%We take the \PYTHIA~6 and \HERWIG{++} difference
%without normalizing them at low \pt. This results in an additional 
This is evaluated to be ${\approx} 7~\%$ 
%as additional uncertainty
for the HP and LP H tag. %, according to Table.~\ref{table:Difference}.
The uncertainty from the pruned jet mass requirement in the $\Hww$ 
search is already included in the extrapolated
scale factor uncertainty of the V-tag. 

The uncertainty in the efficiency of $\Hbb$ tagging can be separated into two categories: the
efficiency related to the b tagging and the efficiency related to the pruned H mass tag. 
The first is
obtained by varying the b tagging scale factors within the associated uncertainties~\cite{BTV-13-001} 
and amounts
to 15\%. The second is assumed 
to be similar to the mass selection efficiency of
W jets estimated in Ref.~\cite{JME-13-006}, additionally accounting
for the difference in fragmentation of light quarks and 
b quarks, which amounts to 2.6\% per jet. 



%Other systematic uncertainties in tagging
%efficiency are even smaller. 
Because of the rejection of charged
particles not originating from the primary vertex, and the application
of pruning, the dependence of the W/Z and H tagging 
efficiencies on
pileup is weak and the uncertainty in the modelling of the pileup
distribution is $\leq 1.5\%$ per jet. 

In this analysis, we only consider $\Hbb$ and $\Hww$ decays. 
Other H decay channels that pass H taggers are viewed
 as nuisance signals, 
and a corresponding cross-talk systematic uncertainty is assigned.
We evaluate this uncertainty as a ratio of expected nuisance signal
events 
with respect to the 
%the expected signals considered in this search, taking into account
total expected signal events, taking into account
the branching fractions, acceptances and tagging efficiencies.  
The contamination from cross-talk is estimated to be $2 - 7\%$
in the \HbbAll\ categories, and $18 - 24\%$ in the \HWWAll\ categories, and 
we take the maximum as the uncertainty. 
The analysis is potentially 7\% (24\%) more sensitive than quoted,
 but since it is not clear how well the efficiency for
 the nuisance signals is understood,
 they are neglected, yielding a conservative limit on new physics. 
When the \HbbAll\ and \HWWAll\ categories are combined together, 
the 24\% uncertainty 
becomes a small effect, based on a quantitative measure of 
sensitivity 
 suggested in Ref.~\cite{punzi} :
\begin{equation}
P = \frac{{\rm \mathcal{B}(H \to XX)} \times \epsilon_{S}}{1+\sqrt{N_B}}
\end{equation}
where ${\rm \mathcal{B}(H \to XX)}$ is the branching fraction for 
the H decay channel, $\epsilon_{S}$ is the signal tagging efficiency, 
and $N_B$ is the corresponding background
yield.
The values of $P$ for each channel are shown in Table~\ref{table:SB}.

\begin{table}[htb]
\begin{center}
  \caption{
%    The pseudo-significances, $S/\sqrt{B}$, for Z' 
%at 1.5 TeV in all five categories, 
%considering luminosity, cross sections, branching fractions, 
%acceptances and tagging efficiencies. Here $S$ represents the expected
% number of tagged signal events, 
%and $B$ represents the number of background dominated data events 
%in the bin of signal resonance mass. 
    Summary of the values $P$ for a Z' signal at 1.5 TeV resonance mass
    and the corresponding background yield 
    in all five categories. 
    \label{table:SB}}
\begin{tabular}{ ccccccc}
\hline
\rule{0pt}{2.4ex} Signal/Categories  & \HbbHP\ & \HbbLP\ & \HWWHP\ & \HWWLPH\ & \HWWLPV\ \\
\hline
\rule{0pt}{2.4ex} \HbbZqq\  & 2.3 $\times 10^{-2}$       &   4.8 $\times 10^{-3}$     &   1.0 $\times 10^{-3}$     & 1.6 $\times 10^{-3}$  & 3.9 $\times 10^{-4}$   \\
\HwwZqq\  & 5.6 $\times 10^{-4}$ & ${\approx} 0$  &  2.6 $\times 10^{-3}$  & 9.8 $\times 10^{-4}$ &  4.5 $\times 10^{-4}$     \\ 
\hline
\end{tabular}
\end{center}
\end{table}





The JES has an uncertainty of
1--2\%~\cite{JME-JINST,Collaboration:2013dp}, and its \pt and $\eta$
dependence is propagated to the reconstructed value of
$m_\mathrm{jj}$, yielding an uncertainty of 1\%, independent
of the resonance mass. The impact of this uncertainty on the
calculated limits is estimated by changing the dijet mass in the
analysis within its uncertainty. The JER is known to a precision of
10\%, and its non-Gaussian features observed in data are well
described by the CMS simulation~\cite{JME-JINST}. The effect of the
JER uncertainty on the limits is estimated by changing the
reconstructed resonance width within its uncertainty. The integrated
luminosity has an uncertainty of 2.6\%~\cite{LUM-13-001}, which is
also taken into account in the analysis. 
%The uncertainty related to
%the PDF used to model the signal acceptance is estimated from the
%eigenvectors of the CT10~\cite{ct10}, MSTW08~\cite{mstw08}, and 
%NNPDF21~\cite{NNPDF}
%PDF sets. 
The uncertainty related to 
the PDF used to model the signal acceptance is estimated from 
the CT10~\cite{ct10}, MSTW08~\cite{mstw08}, and
NNPDF21~\cite{NNPDF} PDF sets.
The envelope of the upward and downward variations of the
estimated acceptance for the three sets is assigned as uncertainty~\cite{Alekhin:2011sk} 
and
found to be 5 -- 15\% in the resonance mass range of interest. A
summary of all systematic uncertainties is given in
Table~\ref{table:systematicsAll} and~\ref{table:systematicsAllCate}. 
Among these uncertainties, the JES
and JER are applied as shape uncertainties, while others 
are applied as uncertainty in the event yield.

\begin{table}[htbp]
\caption{
Systematic uncertainties common to all categories.
}
\begin{center}
\begin{tabular}{ccc}
\hline
Source &  HP uncertainties (\%)   & LP uncertainties (\%) \\  \hline
JES       &   1     &   1 \\ %\hline
JER   & 10      & 10   \\ %\hline
Pileup              & $\leq 3.0$     &  $\leq 3.0$    \\
PDF                     & 5--15   & 5--15  \\ %\hline
Integrated luminosity   & 2.6     &  2.6  \\ 
W-tagging               & 7.5     &  54 \\
W tag \pt dependence     & 4       & 12 \\ \hline
\end{tabular}
\end{center}
\label{table:systematicsAll}
\end{table}


\begin{table}[htbp]
\caption{
Systematic uncertainties(\%) for $ \rm{X \to VH}$ signals, in which 
$\Hbb$ and  $\Hww$.   
Numbers in parentheses represent the 
uncertainty for the corresponding LP category. If LP has the same
uncertainty as HP, only the HP uncertainty is presented here.}
%\begin{tabular}{cc|c|c|c}
\begin{center}
\begin{tabular}{cccc}
\hline
%       &    \multicolumn{3}{c}{Uncertainty (\%) for $ \rm{X \to HV}$ signals, in which H decays to } \\ 
{Source/Final State}     \rule{0pt}{2.2ex}   &    \multicolumn{2}{c}{$\Hbb$} &\multicolumn{1}{c}{$\Hww$}    \\ 
          &  \HbbAll\   & \multicolumn{1}{c}{\HWWAll\ }   & \HWWAll\  \\  \hline

\rule{0pt}{2.4ex} $\Hbb$ mass scale  & 2.6 & - & - \\
H(4q)-tagging  & - & 7.5 (54)            & 7.5 (54) \\
H(4q)-tag $\tau_{42}$ extrapolation & - & 7 & 7 \\
Cross-talk               & 7 & 24       & 24 \\ 
b-tagging & $\leq 15$ & $\leq 15$  & - \\ \hline 
%b-tagging & \multicolumn{2}{c|}{$\leq 15$} & - \\ \hline 
\end{tabular}
\end{center}
\label{table:systematicsAllCate}
\end{table}

