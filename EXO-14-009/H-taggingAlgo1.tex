\section{The H tagging and W/Z tagging algorithms}
\label{sec: H tagging}

%The products of hadronic decays of Higgs, W, and Z bosons can fall
%within a single jet if these particles are highly boosted.  
In this
analysis, we aim to cover as much of the Higgs branching fraction as
possible.  The Standard Model Higgs with a mass of 125 \GeVcc decays
to b$\bar{\rm b}$ with a branching fraction of 57.7\%, and to WW$^*$
with a branching fraction of 21.4\%~\cite{pdg-higgs}.  Using these two
decay modes in a VH search, where WW$^*$ specifically decays to four
quarks, is the main topic of this note. 
% (The semileptonic decay mode
%${\rm H \to WW \to 2q\ell \nu}$ is viable, but its reconstruction is more
%involved and will be covered in a subsequent analysis.)

%% &&& {\bf TO-DO: consider adding the branching fraction plot here. }


The algorithms to identify W/Z, ${\rm H \to b\bar{b}}$ and 
$\Hww$ jets are necessarily different, but they use similar
jet-level variables: N-subjettiness (described in
Section~\ref{sec:N-subjettiness}) and jet pruning
(Section~\ref{sec:jetPruning}).  The W/Z-tagger is described in
Section~\ref{sec:wztagging}, and the two H-taggers in
Sections~\ref{sec:higgsTaggerbb} and~\ref{sec:higgsTaggerww}.


%{\bf TO-DO: add a plot of this Higgs jet mass from signal, QCD, data, as a comparison.  And we need also to show the optimization of this mass choice. }

\subsection{N-subjettiness}
\label{sec:N-subjettiness}

The details of N-subjettiness is elaborated before. Please refer to 
Section~\ref{sec:N-subjettiness1} in Chapter~\ref{chap:chapter3}. 



\subsection{Jet Pruning}
\label{sec:jetPruning}

Please refer to Section~\ref{sec: W/Z tagging1} in 
Chapter~\ref{chap:chapter3} for the procedure of jet pruning. 

Here the result of jet pruning on the CA8 jets is two fold, i.e., the invariant jet mass 
reconstruction and subjet identification.
In all cases,
we use the jet invariant mass computed from the whole (or ``fat'') 
pruned jet.  This quantity is referred below to as the pruned jet mass.
For W/Z tagging, we use pruned jet mass between 70 and 100\GeVcc. 
For the identification of Higgs jets, we require the pruned jet mass to
lie between 110 and 135\GeVcc.  The distribution of the pruned jet
mass of the Higgs candidate jet 
compared to W/Z and top jets is shown on Figure~\ref{fig:JetMassTagging}. 


\begin{figure}[htb]
\begin{center}
%\includegraphics[width=0.49\textwidth]{HbbZqqfigs/Signal/signal-data-qcd-jetmass.pdf}
%\includegraphics[width=0.49\textwidth]{HqqqqZqqfigs/Signal/signal-data-qcd-jetmass.pdf}
%\includegraphics[width=0.49\textwidth]{figs/signal-acc-eff/signal-data-qcd-Jet-Tau21.pdf}
\includegraphics[width=0.69\textwidth]{EXO-14-009/HbbZqqfigs/Signal/signal-data-qcd-jetmass.pdf}
\end{center}
\caption{
Distribution of pruned jet mass in simulation of signal and background processes. All simulated distributions are normalized to 1. The W/Z, H, and top-quark jets are required to match respective generator level particles in the event. The W/Z and H jets are from 
1.5 TeV ${\rm W? \to WH}$ and ${\rm Z? \to ZH}$ signal samples. 
} 
\label{fig:JetMassTagging}
\end{figure}


The main role of jet pruning is to allow better delineation of subjets
within the jet.  In ${\rm H \to b\bar{b}}$ tagger, the axes of the 
pruned subjets are used as the basis for b tagging.



\subsection{W/Z tagging  } 
\label{sec:wztagging}

For the identification of W/Z jets, we employ the same tagging algorithm 
previously used in Section~\ref{sec: W/Z tagging1} in Chapter~\ref{chap:chapter3}.  The W/Z jets are selected using the following requirements:
\begin{itemize}

\item {\bf Pruned jet mass}  $\mbox{\boldmath$m_{\text{jet}}$}$
  - Require the total pruned jet mass to satisfy $70 < m_\text{jet} <  100$\GeVcc.

\item {\bf N-subjettiness} 
  - We split the events into two categories, ``high purity'' $\PW/\cPZ$ jets by
    requiring $\tau_{21} \leq 0.5$, while $ 0.5 < \tau_{21} < 0.75$ defines 
    the ``low purity'' $\PW/\cPZ$ jets.  The thresholds are the same as those in 
    Chapter~\ref{chap:chapter3}. 
\end{itemize}
The performance of the W/Z tagger has been documented in detail in Reference~\cite{JME-13-006}.





%\subsection{Optimization of the Higgs Mass window }







%{\bf TO-DO: add a plot of this Higgs jet mass from signal, QCD, data, as a comparison. 
% Also show the optimization of the resulting choice of the mass window. }

%% &&& {\bf HH to 4b might explain this. }


%\subsection{${\rm H \to b\bar{b}}$}
\subsection{${\rm \Hbb}$ }
\label{sec:higgsTaggerbb}

To identify Higgs jets arising from the shower and hadronization of two 
collimated b quarks, we apply b tagging either on the two subjets or the
fat CA8 jet, based on the angular separation of the two subjets
($\Delta R = \sqrt{ \Delta \eta^2 + \Delta \phi^2 }$)
, which is 
recommended by Reference~\cite{BTV-13-001}. 
So for $\Hbb$ tagging, we use the following selections, synchronized with
the Radion search~\cite{HH4b} and the search for HW 
resonances in the semileptonic channel~\cite{HWlv}:
\begin{itemize}

\item {\bf Pruned jet mass}  $\mbox{\boldmath$m_{\text{jet}}$}$
  - Require the total pruned jet mass to satisfy $110 < m_\text{jet} <  135$\GeVcc.

\item {\bf Subjet b-tagging}
        \begin{itemize}
	\item if $\Delta R$ between the two CA8 subjets is bigger than 0.3: 
  		{\it both} subjets must pass the CSV Loose working point.
	\item if $\Delta R$ between the CA8 subjets is smaller than 0.3:
		require the {\it fat} CA8 jet to pass the CSV Loose working point. 
        \end{itemize}

\end{itemize}



\subsection{${\rm H \to WW^* \to 4q}$}
\label{sec:higgsTaggerww}

%tau4/tau2 < 0.5

In this channel, Higgs decays to two W bosons, one real and one
virtual.  Given that this is effectively a
three-body decay ${\rm H\to Wqq}$, the jets from the four quarks are
not on an even footing -- the subjets from the real W are harder, and
they also form a W mass.  The subjets from the softer two quarks are
less well defined.

A naive ${\rm H \to 4q}$ tagger would require a fat jet with four subjets. 
%{\bf TO-DO: add the plot with the number of subjets.}
However, a study done using the subjets defined by the CMS Top Tagging
algorithm (which reruns the CA8 jet clustering with additional weak 
pruning~\cite{cmstoptagging}) removes $\approx 90\%$ of the signal.   
Compounded with a decreasing angular separation
between Higgs decay products, 
 as a function of the Higgs \pt, at higher 
resonance masses, e.g., at 2\TeVcc, only 1\% of signal 
passes this selection. 
The distribution of the number of subjets of the reconstructed Higgs jets 
in signal MC is shown 
in Figure~\ref{fig:Nsubjets}. 

\begin{figure}[htb]
\centering
\begin{tabular}{cc}
     \resizebox{0.7\linewidth}{!}{\includegraphics{EXO-14-009/HqqqqZqqfigs/N-subjettiness/Nsubjets.pdf}} 
%     \resizebox{0.5\linewidth}{!}{\includegraphics{HqqqqZqqfigs/N-subjettiness/Tau421TeVAfter.pdf}} \\
\end{tabular}
\caption{Number of subjets of the Higgs candidate jets (selected by 
a matching to the generator level Higgs particle), 
in ${\rm W' \to HW}$  signal MC.}
\label{fig:Nsubjets}
\end{figure}


As an alternative, we explore the N-subjettiness, in particular the
variables involving $\tau_4$.  The ratio 
$\tau_{42} \equiv\tau_4/\tau_2$ has the best separation 
between the ${\rm H\to 4q}$
signal and not only QCD background, but also W/Z and top jets.
Figures~\ref{fig:tau421TeV} and~\ref{fig:tau422TeV} show the
discriminating power of $\tau_{42}$ against $\ttbar$ and QCD, for $\rm{m_{V'}}$ at 1~TeV
and 2~TeV resonance masses respectively.

\begin{figure}[htbp]
  \centering
  \begin{tabular}{cc}
    \resizebox{0.5\linewidth}{!}{\includegraphics{EXO-14-009/HqqqqZqqfigs/N-subjettiness/Tau421TeVPre.pdf}} &
    \resizebox{0.5\linewidth}{!}{\includegraphics{EXO-14-009/HqqqqZqqfigs/N-subjettiness/Tau421TeVAfter.pdf}} \\
  \end{tabular}
  \caption{ 
    Distribution for $\tau_{4}/\tau_{2}$ in data and in
    simulations of signal (1.0 TeV) and background events.  All simulated
    distributions are scaled to match the number of events in data,
   % except that matched top is scaled to its fraction of unmatched
   %$ ${\rm t\bar{t}}$ times the number of data events.  
    W/Z, matched
    top and Higgs jets are required to match their generator level
    particles, respectively. }
  \label{fig:tau421TeV}
\end{figure}

\begin{figure}[htbp]
  \centering
  \begin{tabular}{cc}
    \resizebox{0.5\linewidth}{!}{\includegraphics{EXO-14-009/HqqqqZqqfigs/N-subjettiness/tau42PlotAllPre.pdf}} &
    \resizebox{0.5\linewidth}{!}{\includegraphics{EXO-14-009/HqqqqZqqfigs/N-subjettiness/tau42PlotAllAfter.pdf}} \\
  \end{tabular}
  \caption{ 
    Distributions of $\tau_{42}$ in
  data and in simulations of signal (2 TeV) and background events, without applying 
  the pruned jet mass requirement (left) 
  and with the pruned jet mass requirement applied (right). 
  Matched top-quark, W/Z, and ${\rm H_{WW}}$ jets are required to
  be consistent with their generator level particles, respectively.
  All simulated distributions are scaled to the number
  of events in data, except that matched top-quark background is scaled to 
  the fraction of unmatched ${\rm t\bar{t}}$ events times
  the number of data events.
  }
  \label{fig:tau422TeV}
\end{figure}

We also explore other combinations of $\tau_{NM} \equiv \tau_N/\tau_M $,
which are listed in Appendix~\ref{appendix:tauNM}.
The ROC (receiver operating characteristic) 
curve of for several $\tau_{NM}$ cuts (but the same pruned jet mass cut)
is shown in Figure~\ref{fig:roc}.  The signal efficiency is evaluated
using Higgs jets in 2\TeVcc signal MC, and the false positive rate
({\it i.e.}, mistag rate) is derived from QCDPT300to470 MC sample.
From the figure, it is clear that $\tau_{42}$ outperforms any other
single $\tau_{NM}$ variable.
%
%
\begin{figure}[htbp]
  \centering
  \begin{tabular}{cc}
    \resizebox{0.7\linewidth}{!}{\includegraphics{EXO-14-009/ROC.pdf}} 
    %     \resizebox{0.5\linewidth}{!}{\includegraphics{HqqqqZqqfigs/N-subjettiness/Tau421TeVAfter.pdf}} \\
  \end{tabular}
  \caption{ ROC curves for different $\tau_{NM}$ after the cut on the
    pruned jet mass.  The false positive rate (FPR) is obtained from
    QCDPT300to470 and the true positive rate (TPR) from Higgs jets
    in 2 \TeVcc signal MC sample.  Using $\tau_{42}$ to select Higgs
    jets outperforms all other $\tau_{NM}$ variables. }
  \label{fig:roc}
\end{figure}



After optimizing the cut on $\tau_{42}$ (documented in
Section~\ref{sec:tau42Opti} below), the full selection of the ${\rm H \to
  WW^* \to 4q}$ tagger is:
\begin{itemize}

\item {\bf Pruned jet mass}  $\mbox{\boldmath$m_{\text{jet}}$}$
  - We require the total pruned jet mass to satisfy $110< m_\text{jet} <  135$\GeVcc.

\item {\bf N-subjettiness}
  - We split the events into two categories, ``high purity'' Higgs jets by
    requiring $\tau_{42} \leq 0.55$, while $ 0.55 < \tau_{42} < 0.65$ defines
    the ``low purity'' Higgs jets.  

\end{itemize}
 



\clearpage

%\subsection{$\tau_4/\tau_2$ optimization study}
\subsubsection{Optimization of the $\tau_4/\tau_2$ threshold}
\label{sec:tau42Opti}


Having selected $\tau_{42}$ as the discriminating variable, we next
optimize its upper value.  In this study, the jet mass is confined
within $[110, 135]~\GeV$.  We use the limit setting method (described
in Section~\ref{sec:statistics}) and evaluate the expected limits of
several signal resonance masses at different $\tau_{42}$ working
points.  These expected limits are presented
in Table~\ref{table:tau42Opti}.
Given our focus on the resonance masses above 1500~\GeVcc, we
choose to cut on $\tau_{42} < 0.55$. In the following analysis, 
to compensate the signal efficiency loss at higher resonance mass, we 
introduce an additional categories for $\Hww$ tagger as $0.55 < \tau_{42} < 0.65$.
This is chosen from back-of-envelope calculation based on 
Figures~\ref{fig:tau421TeV} and~\ref{fig:tau422TeV}, since this category provides 
very limited sensitivity.  
%Figure~\ref{}i 

\begin{table}[htbp]
\begin{center}
\caption{Upper limits (in units of 0.01~pb) 
for high purity HW and HZ signals at different resonance masses with
different $\tau_{42}$ working points. }
\label{table:tau42Opti}
\begin{tabular}{ccccc}
\hline
HW / $\tau_{42}$ & 0.45 & 0.5 & 0.55 & 0.6 \\ 
1000 & 4.14 & 4.09 & 4.46 & 4.91 \\ 
1500 & 0.97 & 0.88 & 0.86 & 0.91 \\ 
2000 & 0.89 & 0.64 & 0.51 & 0.47 \\ 
2500 & 1.36 & 0.82 & 0.53 & 0.40 \\ \hline 
HZ/ $\tau_{42}$ &   &  & &  \\ 
1000 & 4.31 & 4.36 & 4.63 & 5.05 \\ 
1500 & 0.98 & 0.89 & 0.86 & 0.90 \\ 
2000 & 0.70 & 0.55 & 0.42 & 0.39 \\ 
2500 & 0.96 & 0.61 & 0.41 & 0.32 \\ \hline
\end{tabular}
\end{center}
\end{table}



%\subsection{Summary of Higgs and W/Z tagging categories}
%\label{sec:total}

%In summary, the W or Z jets from the signal are selected by the
%V-tagger, and the Higgs candiadates are selected by an OR of the two
%Higgs taggers, $\Hbb$ and $\Hww$.  Both V-tagger and $\Hww$ taggers 
%have high-purity and
%low-purity categories.  The latter are added to increase the
%sensitivity of the analysis at high resonance masses, where the QCD
%background is low, and a higher signal efficiency is at the premium.
%All the `two-dimensional' categories are shown in
%Table~\ref{table:categories}.  For the \HwwVqq\ channel, we drop the
%low-purity Higgs and low-purity V-tagging category, because it 
%adds only a negligible sensitivity.

%\begin{table}[htb]
%\begin{center}
%  \topcaption{
%    The five event categories used in this analysis.
%    \label{table:categories}}
%\begin{tabular}{ ccc}
%\hline
%$\Hbb$, $\Vqq$ & $\Hww$, $\Vqq$  \\
%\hline
%high-purity V-tag &  high-purity H-tag, high-purity V-tag \\
%low-purity  V-tag &  high-purity H-tag, low purity V-tag\\
%                  &  high-purity V-tag, low purity H-tag\\
%\hline
%\end{tabular}
%\end{center}
%\end{table}




\iffalse

\begin{table}[htbp]
\begin{tabular}{|r|r|r|r|r|r|r|}
\hline
\multicolumn{1}{|l|}{$\tau_{42}$} & \multicolumn{1}{l|}{1000GeV} & \multicolumn{1}{l|}{1500GeV} & \multicolumn{1}{l|}{1800GV} & \multicolumn{1}{l|}{2000GeV} & \multicolumn{1}{l|}{2500GeV} & \multicolumn{1}{l|}{3000GeV} \\ \hline
0.60 & 49.90 & 44.10 & 38.05 & 33.58 & 21.45 & 11.08 \\ \hline
0.55 & 54.20 & 45.33 & 38.52 & 33.91 & 20.88 & 10.69 \\ \hline
0.50 & 55.81 & 45.51 & 37.97 & 32.22 & 19.98 & 9.70 \\ \hline
0.45 & 55.91 & 43.12 & 35.28 & 29.06 & 17.80 & 8.43 \\ \hline
0.40 & 49.17 & 35.99 & 31.21 & 24.55 & 14.13 & 6.73 \\ \hline
0.35 & 37.64 & 28.42 & 22.15 & 18.95 & \multicolumn{1}{l|}{} & \multicolumn{1}{l|}{} \\ \hline
\end{tabular}
\caption{Optimization for $\tau_{42}$ tagger, with a fixed pruned jet 
  mass in $[110~GeV/c^2, 135~GeV/c^2]$.  The table shows the ratio of 
  the number of signal events(without normalization) divided by the number of signal+background
  events(data).}
\label{table:tau42Opti}
\end{table}

\fi



\clearpage

