\chapter{Search for X $\to$ WH or ZH at LHC at $\sqrt{s} = 8$~TeV }
\label{chap:chapter4}
\section{Introduction}
\label{sec:introduction}


Several theories of physics beyond the standard model (SM) predict the
existence of vector resonances with
masses above 1\TeVcc that decay into
a W or Z vector boson (V)
 and a SM-like Higgs boson (H).
Here we present a search for 
the production of such resonances
 in proton-proton (pp) collisions at a
centre-of-mass energy of $\sqrt{s}=8$\TeVcc.  The data sample,
corresponding to an integrated luminosity of 19.7\fbinv, was collected
with the CMS detector at the CERN LHC.


%The following models of ${\rm \PWpr\to HW}$ and ${\rm Z'\to HZ}$ resonances are considered in this analysis.

The composite Higgs~\cite{Composite0,Composite1,Composite2} and little Higgs models~\cite{Han:2003wu}
address the issue of the hierarchy problem and predict many new particles,
%provide a direct solution to the hierarchy problem and 
including additional gauge bosons, e.g. heavy spin-1 $\PWpr$ or ${\rm Z'}$ bosons.
These models can be generalized in the Heavy Vector Triplet (HVT)~\cite{Pappadopulo:2014qza}
framework.
Of particular interest for this search is the HVT scenario B model, where
the branching fraction $\mathcal{B}({\rm W' \to WH})$ and $\mathcal{B}({\rm Z' \to ZH}) $ dominate over the corresponding 
branching fractions to fermions, and are comparable to $\mathcal{B}({\rm W' \to WZ})$ 
and $\mathcal{B}({\rm Z' \to WW})$. 
In this scenario, experimental constraints from searches for boson decay 
channels are more stringent than those from fermion decay channels. 
Several searches~\cite{Khachatryan:2014xja,
Aad:2014pha,Aad:2014xka,ATLASWWPAPER,EXO-12-024} 
for ${\rm W^\prime \rightarrow WZ}$ 
based upon the Extended Gauge Boson (EGB) reference 
model~\cite{egm} have excluded resonance 
masses below 1.7\TeVcc. Unlike the HVT scenario B model, 
the EGB model has enhanced fermionic couplings 
and the mass limit is not directly 
comparable to this work. Model independent limits on 
the cross section for the resonant 
production $\ell\nu+\mathrm{jets}$~\cite{EXO-13-009} can 
be used to extract resonance mass 
limits on the the processes ${\rm W^\prime\rightarrow WZ}$ 
and ${\rm Z^\prime \rightarrow WW}$ of 
1.7 TeV and 1.1 TeV, respectively.
A search for ${\rm Z^\prime \rightarrow ZH \rightarrow q\bar{q}}\tau\tau$
was reported in Ref.~\cite{cms-HZ-tautaujet} and
interpreted in the context of HVT scenario model B; however, 
no resonance mass limit could be set with the sensitivity achieved.
Finally, a recent search~\cite{Aad:2015yza} combining leptonic
decays of W and Z bosons, and two b-tagged jets forming a ${\rm \Hbb}$ candidate
excluded HVT model A with coupling constant $g_V = 1$ for heavy vector
boson masses below $m_{V'^0} < 1360~\GeV$ and $m_{V'^\pm} < 1470~\GeV$.

The signal of interest is a narrow heavy vector resonance ${\rm V'}$ decaying into
VH, where the V decays to a pair of quarks and the H decays either to
a pair of b quarks, or to a pair of W bosons, which further decay into
quarks.
The H in the HVT framework does not have properties that are identical 
to those of a SM Higgs boson. We make the assumption that the state 
observed by the LHC Collaborations~\cite{higgsdiscoveryAtlas,Chatrchyan:2012ufa} 
is the same as the one described by the HVT framework and 
that, in accord with present 
measurements~\cite{Khachatryan:2014kcaCMSHiggs1,Khachatryan:2014jbaCMSHiggs2,AtlasHiggs1}, 
its properties are similar to those of a SM Higgs boson.
  
In the decay of massive ${\rm V'}$ bosons 
produced in the pp collisions at the LHC, 
the momenta of the daughter V and H are large enough (${>} 200\GeV$) 
that their hadronic decay products 
are reconstructed as single jets~\cite{Gouzevitch:2013qca}. 
Because this results in a dijet topology, 
traditional analysis techniques relying on resolved jets are 
no longer applicable. The signal is characterized by a peak 
in the dijet invariant mass ($m_\mathrm{jj}$) distribution 
over a continuous background from mainly QCD multijet 
events. The sensitivity to b-quark jets 
from H decays is enhanced through 
subjet or jet b tagging~\cite{BTV-13-001}. 
Jets from ${\rm \Vqq}$, ${\rm \Hbb}$, and ${\rm \Hww}$ 
(virtual W denoted with an asterisk)
decays are identified with jet 
substructure techniques~\cite{topwtag_pas,JME-13-006}.


This is the first search for heavy resonances 
decaying via VH into all-jet final states 
and it incorporates the first application of jet substructure 
techniques to identify ${\rm \Hww}$ at a high Lorentz boost.


This analysis proceeds via the following steps:
\begin{enumerate}
\item The search is performed in the dijet sample, using the same
      preselection as the standard search for resonances decaying to 
      dijets~\cite{cmsdijet, cmsdijet8TeV}.

\item We identify events with one W/Z boson jet: a candidate
  jet originating from merged decaying products of W/Z:  
  \begin{itemize}
  \item we require a pruned jet mass cut, and
  \item an N-subjettiness cut preferring two-prong decays
  \end{itemize}
  (This is identical to Chapter~\ref{chap:chapter3}.)

\item We identify events with a highly boosted Higgs boson:
  \begin{itemize}
  \item we require a pruned jet mass cut, and
  \item two b tagged subjets, or 
  \item (when there are no two b tagged subjets) a N-subjettiness cut
    preferring four subjets
  \end{itemize}
  (The ${\rm \Hbb}$ tagging is synchronized with our sister analysis, the Radion
  search to the HH final state~\cite{HH4b}.)

\item After the full event selection, a potential signal would be
  characterized as a peak in the dijet invariant mass, on top of a
  falling background distribution.  

\item We model the background distribution with a smoothly falling 
  analytical function.  (The functional form is identical
  to the one used in Chapter~\ref{chap:chapter3}.)

\item We form the joint likelihood of several dijet distributions
  of V tagged and H tagged jets.  We include both two types of
  Higgs tags, and also low-purity Higgs and V taggers.  The background
  estimate procedure is the same in all channels -- analytical
  parametrization -- but is performed separately for each channel.

\item Finally, we set the limits on the various simplified models
  for resonances decaying to HV final states.

\end{enumerate}
