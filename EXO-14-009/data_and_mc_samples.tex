\newpage
%\section{Benchmark model for a vector triplet}
%
%Information about the signal models used  is in Table~\ref{table:signalModels}

%\begin{table}[htbp]
%\resizebox{\textwidth}{!}{
%\begin{tabular}{|r|r|r|r|r|r|r|}
%\hline
%\multicolumn{1}{|l|}{M(GeV)} & \multicolumn{1}{l|}{width(GeV)} & \multicolumn{1}{l|}{BR(Z'$\to$WW)} & \multicolumn{1}{l|}{BR(Z'$\to$HZ ) } & \multicolumn{1}{l|}{BR(W'$\to$ZW)} & \multicolumn{1}{l|}{BR(W'$\to$HW )} & \multicolumn{1}{l|}{X-section(pb)} \\ \hline
%800 & 32.03 & 0.4139 & 0.5672 & 0.4287 & 0.5528 & 3.17E-01 \\ 
%900 & 32.97 & 0.4393 & 0.534 & 0.4513 & 0.5223 & 2.39E-01 \\ 
%1000 & 34.95 & 0.4506 & 0.5176 & 0.4603 & 0.5079 & 1.65E-01 \\ 
%1500 & 47.94 & 0.4665 & 0.4893 & 0.4709 & 0.4849 & 2.44E-02 \\ 
%2000 & 62.2 & 0.4698 & 0.4815 & 0.4723 & 0.4791 & 4.01E-03 \\ 
%2500 & 76.81 & 0.471 & 0.4783 & 0.4726 & 0.4767 & 7.08E-04 \\ 
%3000 & 91.57 & 0.4716 & 0.4765 & 0.4727 & 0.4754 & 1.26E-04 \\ 
%\hline
%\end{tabular}
%}
%\caption{Summary of signal models.}
%\label{table:signalModels}
%\end{table}


\section{Data and Monte Carlo samples}
\label{sec:data_and_mc_samples}


The data sample of proton-proton collisions at $\sqrt{s}=8$~\TeVcc was collected in 2012 and corresponds to
an integrated luminosity of \intlumi.
The datasets and also the certifications
used are summarized in Table~\ref{table:dataset}. 
The dijet sample is dominated by light flavored and gluon jets, which we denote as 
the "QCD background".  
%The QCD background is obtained from data by fitting an
%analytic parameterization of the dijet invariant mass distribution.

We list part of our monte carlo simulated signal(from 1 \TeVcc to 2.6 \TeVcc) 
in 
Table~\ref{table:Hww}. 
Signal samples are generated exclusively of the specific 
Higgs decay mode and W/Z decay mode. 
Model parameters and detailed cross sections are summarized in 
Appendix~\ref{appendix:modelParam}.
The matrix element is calculated with Madgraph~5.1.5.11~\cite{madgraph}. 
% in Table~\ref{table:Hbb} and 
%Table~\ref{table:Hww}, 
The signals of interest,
are showered and hadronized with 
\PYTHIA~6.426~\cite{pythia}, and \HERWIG{++} 2.5.0~\cite{herwig}, 
using simulation of the
CMS detector, based on \GEANTfour~\cite{refGEANT}. Tune
Z2*~\cite{bib_tunez1}
is used in \PYTHIA, while the version 23 tune~\cite{herwig} is used in
\HERWIG{++}. The CTEQ61L~\cite{cteq} parton distribution functions
(PDF) are used for \PYTHIA and the MRST2001~\cite{mrst} leading-order
(LO) PDF for \HERWIG{++}
%All Monte Carlo events are fully simulated and reconstructed via the Geant4-based CMS simulation
% and reconstruction software. 
%Information about the signal models used 
%is in Table~\ref{table:signalModels}
W' and Z' are generated with resonance widths
at $\approx$4\% of the resonance mass, slightly smaller than
the experimental resolution in $m_\mathrm{jj}$ for resonance masses
considered in the analysis. Samples showered from
 \PYTHIA are used in the analysis. %On the other hand, to compare
%the effect of hadronization,  \HERWIG{++} is therefore used
%to retrieve the difference.
While, samples from \HERWIG{++} are used to evaluate the systematic 
uncertainty by comparing the difference of hadronization from \PYTHIA.

%All simulated samples are passed through the standard CMS event
%reconstruction software.



%\begin{table}[htbp]
%\begin{tabular}{|r|r|r|r|r|r|r|}
%\hline
%\multicolumn{1}{|l|}{M(GeV)} & \multicolumn{1}{l|}{width(GeV)} & \multicolumn{1}{l|}{BR(Z'$\to$WW)} & \multicolumn{1}{l|}{BR(Z'$\to$HZ ) } & \multicolumn{1}{l|}{BR(W'$\to$ZW)} & \multicolumn{1}{l|}{BR(W'$\to$HW )} & \multicolumn{1}{l|}{X-section(pb)} \\ \hline
%800 & 32.03 & 0.4139 & 0.5672 & 0.4287 & 0.5528 & 3.17E-01 \\ 
%900 & 32.97 & 0.4393 & 0.534 & 0.4513 & 0.5223 & 2.39E-01 \\ 
%1000 & 34.95 & 0.4506 & 0.5176 & 0.4603 & 0.5079 & 1.65E-01 \\ 
%1500 & 47.94 & 0.4665 & 0.4893 & 0.4709 & 0.4849 & 2.44E-02 \\ 
%2000 & 62.2 & 0.4698 & 0.4815 & 0.4723 & 0.4791 & 4.01E-03 \\ 
%2500 & 76.81 & 0.471 & 0.4783 & 0.4726 & 0.4767 & 7.08E-04 \\ 
%3000 & 91.57 & 0.4716 & 0.4765 & 0.4727 & 0.4754 & 1.26E-04 \\ 
%\hline
%\end{tabular}
%\caption{Summary of signal models.}
%\label{table:signalModels}
%\end{table}



\begin{table}[htb]
\begin{center}
\begin{tabular}{ |l| }
\hline
Dataset                                 \\
\hline
/Jet/Run2012A-22Jan2013-v1/AOD  \\
/JetHT/Run2012B-22Jan2013-v1/AOD  \\
/JetHT/Run2012C-22Jan2013-v1/AOD  \\
/JetHT/Run2012D-22Jan2013-v1/AOD  \\
\hline
\end{tabular} 
\end{center}
\caption{Summary of 8~\TeVcc collision data used in this analysis. 
The certification file used for these data is 
{\tt Cert\_190456-208686\_8TeV\_22Jan2013ReReco\_Collisions12\_JSON.txt
}.
}
\label{table:dataset}
\end{table}

\begin{table}[htb]
\begin{center}
\begin{tabular}{ |l|c|r|r| }
\hline
Process     & mass ($\GeVcc$) & Events & X-sec[pb] \\
\hline
Z' $\to$ HZ & 1000   &20000   & 8.56E-02 \\
 & 1500   &20000              & 1.19E-02 \\
 & 2000   &20000              & 1.93E-03 \\
 & 2500  &20000               & 3.39E-04  \\\hline
%W' $\to$ H(ww $\to$ qqqq)W(qq)(m=750$\GeVcc$) &Madgraph   &20000   &4.071E+01  \\
W' $\to$ HW& 1000   &20000   &  1.71E-01  \\
 & 1500 &20000               &  2.55E-02  \\
 & 2000 &20000               &  4.25E-03  \\
 & 2500  &20000              &  7.31E-04  \\
\hline
\end{tabular}
\end{center}
\caption{Examples of the simulated Monte Carlo samples used in this analysis for process
 V' $\to$ VH. Cross sections are calculated from 
production cross sections of V' times its BR(W' $\to$HW or Z' $\to$ HZ). 
 These samples are generated using Madgrap5 and hadronized with Pythia6. }
\label{table:Hww}
\end{table}


\iffalse

\begin{table}[htb]
\begin{center}
\begin{tabular}{ |l|c|r|r| }
\hline
Process           & mass ($\GeVcc$) & Events & X-sec[pb] \\
\hline
Z' $\to$ H(bb)Z(qq) & 1000  &20000   & 3.45E-02 \\
 & 1500     &20000   & 4.81E-03 \\
 & 2000    &20000   & 7.79E-04  \\
 & 2500    &20000   & 1.37E-04 \\\hline
%W' $\to$ H(bb)W(qq)(m=750$\GeVcc$) &Madgraph   &20000   &4.071E+01  \\
W' $\to$ H(bb)W(qq)& 1000  &20000   & 3.28E-02 \\
 & 1500    &20000   & 4.61E-03 \\
 & 2000    &20000   & 7.50E-04 \\
 & 2500     &20000   & 1.32E-04 \\
\hline
\end{tabular}
\end{center}
\caption{Examples of the simulated Monte Carlo samples used in this analysis for process
 Z'/W'$\to$Z/W(qq) + H(bb). Those samples was generated using Madgrap5 and hadronized with Pythia6.}
\label{table:Hbb}
\end{table}

\begin{table}[htb]
\begin{center}
\begin{tabular}{ |l|c|r|r| }
\hline
Process     & mass ($\GeVcc$) & Events & X-sec[pb] \\
\hline
Z' $\to$ H(ww $\to$ qqqq)Z(qq) & 1000   &20000   & 5.88E-03 \\
 & 1500   &20000   & 8.19E-04 \\
 & 2000   &20000   & 1.33E-04 \\
 & 2500  &20000   & 2.33E-05 \\\hline
%W' $\to$ H(ww $\to$ qqqq)W(qq)(m=750$\GeVcc$) &Madgraph   &20000   &4.071E+01  \\
W' $\to$ H(ww $\to$ qqqq)W(qq)& 1000   &20000   & 5.58E-03 \\
 & 1500 &20000   & 7.85E-04  \\
 & 2000 &20000   & 1.28E-04 \\
 & 2500  &20000   & 2.24E-05 \\
\hline
\end{tabular}
\end{center}
\caption{Examples of the simulated Monte Carlo samples used in this analysis for process
 Z'/W'$\to$Z/W(qq) + H(ww $\to$ qqqq). Those samples was generated using Madgrap5 and hadronized with Pythia6.}
\label{table:Hww}
\end{table}

\fi


