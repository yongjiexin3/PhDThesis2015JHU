\newpage

\section{Data and Monte Carlo samples}
\label{sec:data_and_mc_samples}


The data sample of proton-proton collisions at $\sqrt{s}=8$~\TeVcc was collected in 2012 and corresponds to
an integrated luminosity of \intlumi. It is the same as the data studied in 
Chapter~\ref{chap:chapter3} Table~\ref{table:dataset}. 
The dijet data sample is dominated by light flavored and gluon jets, which we denote as 
the ``QCD background".  

In the HVT framework, the production cross sections of 
${\rm W'}$ and ${\rm Z'}$ bosons and their decay branching fractions depend on 
three parameters in addition to the resonance masses:  the 
strength of couplings to quarks ($c_\mathrm{q}$), to the H ($c_\mathrm{H}$),  and on
their self-coupling ($g_\mathrm{V}$).
In the HVT model B, 
where $g_\mathrm{V}=3$ and $c_\mathrm{q}=-c_\mathrm{H}=1$, 
 ${\rm W'}$ and ${\rm Z'}$ preferentially couple to bosons (W/Z/H),
giving rise to diboson final states.
This feature reproduces the properties of the ${\rm W'}$ and ${\rm Z'}$ bosons
 predicted by the minimal composite Higgs model.
In this case, the production cross sections for ${\rm Z'}$,
${\rm W'^{-}}$, and ${\rm W'^{+}}$ are respectively 165, 
87, and 248 fb
for a signal of resonance mass $m_\mathrm{V'} =$ 1 TeV.
Their branching
fractions to VH and decay width are respectively   
51.7\%, 50.8\%, 50.8\% and 35.0, 34.9, 34.9 GeV. 
The resonances are assumed to be narrow, {\it i.e.}, with natural
widths smaller than the experimental resolution in $m_\mathrm{jj}$ for
masses considered in this analysis.

We consider the ${\rm W'}$ and ${\rm Z'}$ 
resonances separately, and report limits for 
each candidate individually to permit 
the reinterpretation of our results in 
different scenarios with different numbers of spin-1 resonances.

Signal events are simulated using 
the \MADGRAPH~5.1.5.11~\cite{madgraph} Monte Carlo (MC) event 
generator to generate partons that are 
then showered with with \PYTHIA~6.426~\cite{Sjostrand:2006za} to 
produce final state particles. These events are then 
processed through a \GEANTfour~\cite{refGEANT} based simulation of 
the CMS detector. The \MADGRAPH input parameters are 
provided in Ref.~\cite{Pappadopulo:2014qza1} and the H mass 
is assumed to be 125\GeVcc. Samples showered 
with \HERWIG{++}~2.5.0~\cite{herwig} are 
used to evaluate the systematic uncertainty 
associated with the hadronization. Tune Z2*~\cite{bib_tunez1} is 
used in \PYTHIA, while the version 23 tune~\cite{herwig} is 
used in \HERWIG{++}. The CTEQ6L1~\cite{cteq} parton distribution 
functions (PDF) are used for \MADGRAPH, \PYTHIA and \HERWIG{++}. 
Signal events are generated from resonance mass 1.0 to 2.6 TeV in 
steps of 0.1 TeV. Signals with resonance masses 
between the generated values are interpolated.
Part of the signal samples and their cross sections are listed in Table~\ref{table:Hww}.


\begin{table}[htb]
\begin{center}
\begin{tabular}{ cccc }
\hline
Process     & mass (\GeVcc) & Events & X-sec[pb] \\
\hline

Z' $\to$ HZ & 1000   &20000   & 8.56E-02 \\
 & 1500   &20000              & 1.19E-02 \\
 & 2000   &20000              & 1.93E-03 \\
 & 2500  &20000               & 3.39E-04  \\\hline

W' $\to$ HW& 1000   &20000   &  1.71E-01  \\
 & 1500 &20000               &  2.55E-02  \\
 & 2000 &20000               &  4.25E-03  \\
 & 2500  &20000              &  7.31E-04  \\
\hline
\end{tabular}
\end{center}
\caption{Examples of the simulated Monte Carlo samples used in this analysis for process
 ${\rm V' \to VH}$. Cross sections are calculated by 
its production cross sections of ${\rm V'}$ times its $\mathcal{B}$( ${\rm W' \to HW}$ or ${\rm Z' \to HZ}$). }
\label{table:Hww}
\end{table}


\iffalse

\begin{table}[htb]
\begin{center}
\begin{tabular}{ |l|c|r|r| }
\hline
Process           & mass ($\GeVcc$) & Events & X-sec[pb] \\
\hline
Z' $\to$ H(bb)Z(qq) & 1000  &20000   & 3.45E-02 \\
 & 1500     &20000   & 4.81E-03 \\
 & 2000    &20000   & 7.79E-04  \\
 & 2500    &20000   & 1.37E-04 \\\hline
%W' $\to$ H(bb)W(qq)(m=750$\GeVcc$) &Madgraph   &20000   &4.071E+01  \\
W' $\to$ H(bb)W(qq)& 1000  &20000   & 3.28E-02 \\
 & 1500    &20000   & 4.61E-03 \\
 & 2000    &20000   & 7.50E-04 \\
 & 2500     &20000   & 1.32E-04 \\
\hline
\end{tabular}
\end{center}
\caption{Examples of the simulated Monte Carlo samples used in this analysis for process
 Z'/W'$\to$Z/W(qq) + H(bb). Those samples was generated using Madgrap5 and hadronized with Pythia6.}
\label{table:Hbb}
\end{table}

\begin{table}[htb]
\begin{center}
\begin{tabular}{ |l|c|r|r| }
\hline
Process     & mass ($\GeVcc$) & Events & X-sec[pb] \\
\hline
Z' $\to$ H(ww $\to$ qqqq)Z(qq) & 1000   &20000   & 5.88E-03 \\
 & 1500   &20000   & 8.19E-04 \\
 & 2000   &20000   & 1.33E-04 \\
 & 2500  &20000   & 2.33E-05 \\\hline
%W' $\to$ H(ww $\to$ qqqq)W(qq)(m=750$\GeVcc$) &Madgraph   &20000   &4.071E+01  \\
W' $\to$ H(ww $\to$ qqqq)W(qq)& 1000   &20000   & 5.58E-03 \\
 & 1500 &20000   & 7.85E-04  \\
 & 2000 &20000   & 1.28E-04 \\
 & 2500  &20000   & 2.24E-05 \\
\hline
\end{tabular}
\end{center}
\caption{Examples of the simulated Monte Carlo samples used in this analysis for process
 Z'/W'$\to$Z/W(qq) + H(ww $\to$ qqqq). Those samples was generated using Madgrap5 and hadronized with Pythia6.}
\label{table:Hww}
\end{table}

\fi


