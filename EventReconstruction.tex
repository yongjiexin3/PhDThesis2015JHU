\section{Event reconstruction and selections}
\label{label:EventsReconstruction}

In proton-proton (pp) collisions at the energies reached at the LHC,  initials partons smashed to radiate quarks and gluons. As quarks and gluons have a net colour charge and cannot exist freely due to colour-confinement, they are not directly observed in Nature. 
Instead, they come together to form colour-neutral hadrons, a process called hadronisation that leads to a collimated spray of hadrons called a jet.
